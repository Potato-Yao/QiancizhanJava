\section[解析器和管理器]{编写解析器和管理器}\label{sec:解析器和管理器}

解析器用于解析单词本文件,它将单词本文件的单词、历史记录和单词本信息分别解析为\emph{List<Word>}、\emph{List<History>}和\emph{Info}。管理器则是用于单词本的增、删、查、改。

\subsection[解析器]{编写解析器}\label{subsec:解析器}

编写解析器,首先需要继承\emph{Parser}类并实现构造器和\emph{parser()}方法。

\begin{lstlisting}[style=Java, caption={初始布局},label={lst:初始布局}]
import com.potato.Parser.Parser;
import java.io.File;

public class MyParser extends Parser
{
    public MyParser(File file, String extension)
    {
        super(file, extension);
    }

    @Override
    protected void parser()
    {

    }
}
\end{lstlisting}

由于我们自定义的解析器对应的文件类型一定是确定的,因此我们应该将构造器中的\emph{String extension}换成我们解析器对应的文件类型。所有单词本文件的文件类型都定义在了\emph{com.potato.ToolKit.WordFileType}中,我们以数据库为例,构造器更改如下所示:

\begin{lstlisting}[style=Java, caption={构造器},label={lst:构造器}]
public MyParser(File file)
{
    super(file, WordFileType.DATABASE.type());
}
\end{lstlisting}

接下来编写\emph{parser()}方法。这个方法是解析单词本文件的具体实现,它是一个抽象方法(因此必须实现),将直接被\emph{Parser}的构造器调用。因此如果有需要在解析前初始化的对象,需要写在构造器中。

\begin{lstlisting}[style=Java, caption={Parser的构造器源码},label={lst:Parser的构造器源码}]
/**
 * Parser用于解析储存单词本的文件
 *
 * @param file       需要解析的文件
 * @param extension 文件应有的扩展名,对于文件xxx.db,其扩展名是db
 *                  扩展名建议使用WordFileType中已定义的枚举
 */
public Parser(File file, String extension)
{
    // 使用卫语句捕捉文件类型错误
    if (!extension.equals(getExtensionName(file)))
    {
        Log.e(getClass().toString(), String.format("解析文件%s类型错误", file.getName()));
    }
    this.file = file;

    parser();
    Log.i(getClass().toString(), String.format("解析文件%s成功", file.getName()));
}
\end{lstlisting}

\emph{Parser}中定义了三个静态变量:\emph{wordList}、\emph{info}和\emph{historyList},它们分别储存单词列表、单词本信息和历史记录(单词本文件的详细说明和格式规范、单词的构造方法、历史记录和数据库的构造器请见API指南)。这三个静态变量都给出了对应的Setter,\emph{parser()}方法的作用就是给这三个变量赋值。

\begin{lstlisting}[style=Java, caption={parser方法},label={lst:parser方法}]
@Override
protected void parser()
{
    List<Word> wordList = new ArrayList<>();
    List<History> historyList = new ArrayList<>();
    Info info = null;

    // 对以上三者赋值
    // ...

    setWordList(wordList);
    setHistoryList(historyList);
    setInfo(info);
}
\end{lstlisting}

至此就完成了解析器的编写。我们在调用\emph{AutoParser}时,它会从\emph{Config.parserMap}中选取对应文件类型的解析器,因此我们要在这里更改解析器为我们的解析器。

\emph{Config}提供了\emph{setParser(WordFileType, Constructor<? extends Parser>)}方法用于修改某文件类型对应的解析器。修改解析器建议在\emph{Config}初始化时就进行,以免因为程序执行顺序错误而选用了错误了解析器。

\begin{lstlisting}[style=Java, caption={修改解析器},label={lst:修改解析器}]
// getConstructor()的参数是MyParser构造器中每个参数对应类型的class
Config.setParser(WordFileType.DATABASE, MyParser.class.getConstructor(File.class));
\end{lstlisting}

\subsection[管理器]{编写管理器}\label{subsec:管理器}
